\documentclass[parskip=full]{scrartcl}
\usepackage[utf8]{inputenc} % use utf8 file encoding for TeX sources
\usepackage[T1]{fontenc}    % avoid garbled Unicode text in pdf
\usepackage[german]{babel}  % german hyphenation, quotes, etc
\usepackage{hyperref}       % detailed hyperlink/pdf configuration
\hypersetup{                % ‘texdoc hyperref‘ for options
pdftitle={Lineare Algebra - Zusammenfassung},%
bookmarks=true,%
}
\usepackage{graphicx}       % provides commands for including figures
\usepackage{csquotes}       % provides \enquote{} macro for "quotes"
\usepackage[nonumberlist]{glossaries}     % provides glossary commands
\usepackage{enumitem}
\usepackage{eurosym}
\usepackage{graphicx}
\usepackage{subcaption}
\usepackage[section]{placeins}
\graphicspath{ {./images/} }
\usepackage{pxfonts}
\usepackage{mathtools}

\title{Lineare Algebra - Eine informative Einleitung und Zusammenfassung}
\subtitle{Dieses Dokument stellt die lineare Algebra aus der sicht eines Informatikers dar}
\author{Alexander Sommer}

\begin{document}

\maketitle

\section{Einleitung}
    Die Lineare Algebra beschäftigt sich u.a. mit der formalen Definition mathematischer Muster.
    Durch Abstraktion können Aussagen und Gebilde akurat und kompakt dargestellt werden.
    \\Die am Anfang oft komplizert aussehenden Konstruktionen und Definitionen stellen sich später
    (im Verlauf des Studiums) aber doch als logisch und sinnvoll heraus.

\section{Mengen}
    Die lineare Algebra bzw. die moderne Mathematik baut auf der Mengenlehre auf.
    Prinzipiell ist eine Menge eine Ansammlung von \textbf{ungeordneten} Objekten, die nur einmal in dieser vorkommen können.
    Also kann ein Objekt in einer Menge entweder (einmal) enthalten sein oder nicht.
    \\Aber was ist denn nun ein solches Objekt?
    \\Naja prinzipiell begrenzt die lineare Algebra dies nicht.
    Sinnvoll für die Mathematik wäre nur, wenn man mit diesem auch irgendwie \enquote{rechnen} kann. 
    Zahlen sind hier naheliegend, aber später werden noch andere Objekte auftauchen mit denen man auch \enquote{rechnen} kann.
    \\Wir halten also fest, das eine Menge X eine lose Ansammlung von Objekten ist, 
    aus der man sich Elemente herausnehmen kann bzw. entscheiden kann, ob ein Element x darin enthalten ist oder nicht (\(x \in X\) oder \(x\notin X\)).
    
    \subsection{Mengenoperationen- und relationen}
        Manchmal möchte man gegebene Mengen manipulieren, um eine neue Menge zu erhalten und mit dieser weiter zu arbeiten.
        \\So gibt es die Vereinigung \(\cup\), den Schnitt \(\cap\) und die Differenz \(\setminus\) von Mengen. 
        Bei der Mengendifferenz \(A \setminus B\) werden alle Elemente die in A und B enthalten sind herausgenommen
        und es bleiben die Elemente zurück, die nur in A enthalten sind.
        \\Die Vereinigung und den Schnitt kann man sich als Informatiker relativ einfach merken und herleiten:
        Die Symbole für diese Operationen ähneln nämlich den logischen Operatoren \(\wedge\) (UND) und \(\vee\) (ODER).
        So ist ein Element was in \(A \cap B\) enthalten ist, in A \textbf{und} in B (also in beiden) enthalten. 
        Währendessen ein Element aus \(A \cup B\) entweder in A \textbf{oder} in B \textbf{oder} in beiden enthalten.
        \\Hier sieht man schon eine erste mögliche Beziehung zwischen Mengen: 
        Es gilt nämlich \(A \cap B\) ist \textbf{Teilmenge} von \(A \cup B\) \quad (\(A \cap B \subseteq A \cup B\)).
        Wenn ein Element im Schnitt enthalten ist, ist es nämlich insbesondere in der Vereinigung enthalten (Warum?).
        \\Eine weitere Relation ist die Gleichheit von Mengen: Zwei Mengen sind genau dann gleich, wenn sie die gleichen Elemente enthalten.
        Für Beweise ist es jedoch sinnvoller die Folgende Definition zu nutzen: 
        Zwei Mengen sind gleich, wenn sie jeweils Teilmengen voneinander sind \quad \(A = B \iff A \subseteq B \wedge B \subseteq A\).
        
    \subsection{Zahlenmengen}
        Die elementarste Zahlenmenge ist wohl die der natürlichen Zahlen \(\mathbb{N} := \{1, 2, 3, \dots\}\). 
        \\Durch hinzunehmen der Null entsteht \(\mathbb{N}_0 := \mathbb{N} \cup \{0\}\).
        \\Die ganzen Zahlen bekommt man durch hinzunehmen der negativen natürlichen Zahlen \(\mathbb{Z} := \mathbb{N}_0 \cup \{-n : n \in \mathbb{N}\}\)
        \\Die rationalen Zahlen werden als Brüche ganzer Zahlen dargestellt
        \\\(\mathbb{Q} := \{\frac{a}{b} : a \in \mathbb{Z}, b \in \mathbb{N}\}\)
        \quad(Man beachte dabei, das nicht durch null geteilt werden darf!).
        \\Die reelen Zahlen \(\mathbb{R}\) lassen sich leider nicht so schön aufbauend herleiten.
        Es sei aber gesagt, dass diese sich aus den rationalen Zahlen \(\mathbb{Q} \subseteq \mathbb{R}\) und den irrationalen Zahlen \(\mathbb{R} \setminus \mathbb{Q}\) aufbauen.
        Irrationale Zahlen sind dann z.B. \(\pi, \mathrm{e}\) oder \(\sqrt{2}\).

\section{Relationen und Abbildungen}
    \subsection{Relationen}
        Oben haben wir mögliche Beziehungen zwischen Mengen gesehen.
        Nun wollen wir uns Beziehungen einzelner Elemente von Mengen anschauen: Die sogenannten Relationen.
        Eine Relation sagt verallgemeinert aus, ob ein Element mit einem anderen in einer bestimmten Weise in Beziehung steht oder nicht.
        Das schöne daran ist nun, das wir Relationen wieder per Definition auf die Mengenlehre zurückführen können.

        Wir führen zunächst das kartesische Produkt zweier Mengen A und B ein:
        \\\quad\(A \times B := \{(a, b) : a \in A, b \in B\}\)
        \\Welche Elemente kann man nun aus dieser neuen Menge \(A \times B\) herausnehmen?
        Im kartesischen Produkt finden wir jede Kombination von den Elementen aus A mit Elementen aus B.
        Ein Element aus diesem kartesischen Produkt nennen wir (2-)Tupel oder auch Paar. 
        Ein Tupel ist eine Liste oder Folge von Elementen aus Mengen. 
        In unserem Fall ist der erste Eintrag aus der Menge A und der zweite Eintrag aus der Menge B.

        Wenn wir uns jetzt nochmal die Relationen von oben anschauen erkennen wir, 
        das eine Relation R von Elementen von zwei Mengen A und B nichts anderes ist als eine Teilmenge des kartesischen Produkts
        (\(R \subseteq A \times B\)).
        Die Elemente \(a \in A\) und \(b \in B\) stehen also genau dann in Relation (\(a \sim b\)), 
        wenn sie in der Menge \(R \subseteq A \times B\), die die Relation beschreibt, als Tupel enthalten sind
        (\(a \sim b \iff (a, b) \in R\)).

        Ein einfaches Beispiel ist die Kleiner-Gleich-Relation \(R_\leq\) als Teilmenge von \(\mathbb{N} \times \mathbb{N}\), 
        bei der für ein Element \((a, b) \in R_\leq\) stets gilt, das a kleiner oder gleich b ist 
        \\(\((a, b) \in R_\leq \subseteq \mathbb{N} \times \mathbb{N} \iff a \leq b\)). 

    \subsection{Abbildungen}

\section{Gruppen}
    \subsection{Verknüpfungen}

\end{document}